\documentclass[]{article}
\usepackage[left=1in,top=1in,right=1in,bottom=1in]{geometry}
\newcommand*{\authorfont}{\fontfamily{phv}\selectfont}
\usepackage{lmodern}


  \usepackage[T1]{fontenc}
  \usepackage[utf8]{inputenc}



\usepackage{abstract}
\renewcommand{\abstractname}{}    % clear the title
\renewcommand{\absnamepos}{empty} % originally center

\renewenvironment{abstract}
 {{%
    \setlength{\leftmargin}{0mm}
    \setlength{\rightmargin}{\leftmargin}%
  }%
  \relax}
 {\endlist}

\makeatletter
\def\@maketitle{%
  \newpage
%  \null
%  \vskip 2em%
%  \begin{center}%
  \let \footnote \thanks
    {\fontsize{18}{20}\selectfont\raggedright  \setlength{\parindent}{0pt} \@title \par}%
}
%\fi
\makeatother




\setcounter{secnumdepth}{0}



\title{Dissertation drafts and notes }
 



\author{\Large \vspace{0.05in} \newline\normalsize\emph{}  }


\date{}

\usepackage{titlesec}

\titleformat*{\section}{\normalsize\bfseries}
\titleformat*{\subsection}{\normalsize\itshape}
\titleformat*{\subsubsection}{\normalsize\itshape}
\titleformat*{\paragraph}{\normalsize\itshape}
\titleformat*{\subparagraph}{\normalsize\itshape}

\newcommand{\dummy}[1]{#1}

\usepackage{natbib}
\bibpunct{(}{)}{;}{a}{}{,}
\bibliographystyle{bath}
%\usepackage[strings]{underscore} % protect underscores in most circumstances



\newtheorem{hypothesis}{Hypothesis}
\usepackage{setspace}

\makeatletter
\@ifpackageloaded{hyperref}{}{%
\ifxetex
  \PassOptionsToPackage{hyphens}{url}\usepackage[setpagesize=false, % page size defined by xetex
              unicode=false, % unicode breaks when used with xetex
              xetex]{hyperref}
\else
  \PassOptionsToPackage{hyphens}{url}\usepackage[unicode=true]{hyperref}
\fi
}

\@ifpackageloaded{color}{
    \PassOptionsToPackage{usenames,dvipsnames}{color}
}{%
    \usepackage[usenames,dvipsnames]{color}
}
\makeatother
\hypersetup{breaklinks=true,
%            bookmarks=true,
            pdfauthor={ ()},
             pdfkeywords = {},  
            pdftitle={Dissertation drafts and notes},
            colorlinks=true,
            citecolor=blue,
            urlcolor=blue,
            linkcolor=magenta,
            pdfborder={0 0 0}}
\urlstyle{same}  % don't use monospace font for urls

% set default figure placement to htbp
\makeatletter
\def\fps@figure{htbp}
\makeatother



% add tightlist ----------
\providecommand{\tightlist}{%
\setlength{\itemsep}{0pt}\setlength{\parskip}{0pt}}

\begin{document}
	
% \pagenumbering{arabic}% resets `page` counter to 1 
%    

% \maketitle

{% \usefont{T1}{pnc}{m}{n}
\setlength{\parindent}{0pt}
\thispagestyle{plain}
{\fontsize{18}{20}\selectfont\raggedright 
\maketitle  % title \par  

}

{
   \vskip 13.5pt\relax \normalsize\fontsize{11}{12} 
\textbf{\authorfont } \hskip 15pt \emph{\small }   

}

}






\vskip 6.5pt


\noindent  \hypertarget{introduction}{%
\section{Introduction}\label{introduction}}

\hypertarget{sexual-selection-theory-and-mating-strategies}{%
\subsection{Sexual selection theory and mating
strategies}\label{sexual-selection-theory-and-mating-strategies}}

\hypertarget{the-darwin-bateman-paradigm}{%
\subsubsection{The Darwin-Bateman
paradigm}\label{the-darwin-bateman-paradigm}}

Until a relatively recent diversification of approaches, sexual
selection theory has long been used to assert that differences in mating
strategies between the sexes arise from a fundamental dichotomy between
``coy females'' and ``promiscuous males'' - sometimes characterised as
the ``Darwin-Bateman paradigm'' \citep{borgerhoffmulder2021}. As the
name would suggest, it is rooted in \citet{darwin1981} 's early
assertion that males in almost all animal species are more eager and
less discriminate than females when it comes to reproduction. This,
Darwin continues, makes them the primary target of evolution via sexual
selection - either through male-male competition or through female
choice, the inverse processes of which (e.g.~female-female competition,
etc.) he largely neglects. Darwin, of course, could only speculate about
the origin of these sex differences, which he tentantively reasoned to
be driven by the fact that insemination requires males to closely
approach females, thus selecting for an ``eagerness'' in males to seek
out females \citep{darwin1981}. But the brevity with which this,
entirely non-empirical, explanation is offered in The Descent of Man
perhaps speaks to the lack of confidence Darwin had in this line of
reasoning. Darwin's ideas on sexual selection and the associated coy
female-ardent male dichotomy largely languished in scientific no-man's
land until the latter half of the eponymous paradigm, Angus J. Bateman,
published an article titled ``Intra-sexual selection in Drosophila'' in
Heredity in 1948. Here, \citet{bateman1948}

\hypertarget{explanations-for-marriage-patterns-in-humans}{%
\subsection{Explanations for marriage patterns in
humans}\label{explanations-for-marriage-patterns-in-humans}}

\hypertarget{polygyny-threshold-model}{%
\subsubsection{Polygyny threshold
model}\label{polygyny-threshold-model}}

\hypertarget{harem-defense-polygyny}{%
\subsubsection{Harem defense polygyny}\label{harem-defense-polygyny}}

\hypertarget{market-forces-sex-ratio-dependent-polygyny}{%
\subsubsection{Market forces / sex-ratio dependent
polygyny}\label{market-forces-sex-ratio-dependent-polygyny}}

\hypertarget{rival-wealth-induced-monogamy}{%
\subsubsection{Rival wealth-induced
monogamy}\label{rival-wealth-induced-monogamy}}

\hypertarget{cultural-group-selection}{%
\subsubsection{Cultural group
selection}\label{cultural-group-selection}}




\newpage
\singlespacing 
\bibliography{\dummy{C:/Users/fergu/Documents/msc_diss/litreview/Dissertation.bib}}

\end{document}