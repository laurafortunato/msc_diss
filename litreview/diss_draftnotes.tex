% Options for packages loaded elsewhere
\PassOptionsToPackage{unicode}{hyperref}
\PassOptionsToPackage{hyphens}{url}
%
\documentclass[
]{article}
\usepackage{amsmath,amssymb}
\usepackage{lmodern}
\usepackage{ifxetex,ifluatex}
\ifnum 0\ifxetex 1\fi\ifluatex 1\fi=0 % if pdftex
  \usepackage[T1]{fontenc}
  \usepackage[utf8]{inputenc}
  \usepackage{textcomp} % provide euro and other symbols
\else % if luatex or xetex
  \usepackage{unicode-math}
  \defaultfontfeatures{Scale=MatchLowercase}
  \defaultfontfeatures[\rmfamily]{Ligatures=TeX,Scale=1}
\fi
% Use upquote if available, for straight quotes in verbatim environments
\IfFileExists{upquote.sty}{\usepackage{upquote}}{}
\IfFileExists{microtype.sty}{% use microtype if available
  \usepackage[]{microtype}
  \UseMicrotypeSet[protrusion]{basicmath} % disable protrusion for tt fonts
}{}
\makeatletter
\@ifundefined{KOMAClassName}{% if non-KOMA class
  \IfFileExists{parskip.sty}{%
    \usepackage{parskip}
  }{% else
    \setlength{\parindent}{0pt}
    \setlength{\parskip}{6pt plus 2pt minus 1pt}}
}{% if KOMA class
  \KOMAoptions{parskip=half}}
\makeatother
\usepackage{xcolor}
\IfFileExists{xurl.sty}{\usepackage{xurl}}{} % add URL line breaks if available
\IfFileExists{bookmark.sty}{\usepackage{bookmark}}{\usepackage{hyperref}}
\hypersetup{
  pdftitle={Dissertation drafts and notes},
  pdfauthor={Sven Kasser},
  hidelinks,
  pdfcreator={LaTeX via pandoc}}
\urlstyle{same} % disable monospaced font for URLs
\usepackage[margin=1in]{geometry}
\usepackage{graphicx}
\makeatletter
\def\maxwidth{\ifdim\Gin@nat@width>\linewidth\linewidth\else\Gin@nat@width\fi}
\def\maxheight{\ifdim\Gin@nat@height>\textheight\textheight\else\Gin@nat@height\fi}
\makeatother
% Scale images if necessary, so that they will not overflow the page
% margins by default, and it is still possible to overwrite the defaults
% using explicit options in \includegraphics[width, height, ...]{}
\setkeys{Gin}{width=\maxwidth,height=\maxheight,keepaspectratio}
% Set default figure placement to htbp
\makeatletter
\def\fps@figure{htbp}
\makeatother
\setlength{\emergencystretch}{3em} % prevent overfull lines
\providecommand{\tightlist}{%
  \setlength{\itemsep}{0pt}\setlength{\parskip}{0pt}}
\setcounter{secnumdepth}{-\maxdimen} % remove section numbering
\ifluatex
  \usepackage{selnolig}  % disable illegal ligatures
\fi
\usepackage[]{natbib}
\bibliographystyle{bath}

\title{Dissertation drafts and notes}
\author{Sven Kasser}
\date{06/07/2021}

\begin{document}
\maketitle

\hypertarget{introduction}{%
\section{Introduction}\label{introduction}}

\hypertarget{sexual-selection-theory-and-mating-strategies}{%
\subsection{Sexual selection theory and mating
strategies}\label{sexual-selection-theory-and-mating-strategies}}

\hypertarget{the-darwin-bateman-paradigm}{%
\subsubsection{The Darwin-Bateman
paradigm}\label{the-darwin-bateman-paradigm}}

Until a relatively recent diversification of approaches, sexual
selection theory has long been used to assert that differences in mating
strategies between the sexes arise from a fundamental dichotomy between
``coy females'' and ``promiscuous males'' - sometimes characterised as
the ``Darwin-Bateman paradigm'' \citep{borgerhoffmulder2021}. As the
name would suggest, it is rooted in \citet{darwin1981} 's early
assertion that males in almost all animal species are more eager and
less discriminate than females when it comes to reproduction. This,
Darwin continues, makes them the primary target of evolution via sexual
selection - either through male-male competition or through female
choice, the inverse processes of which (e.g.~female-female competition,
etc.) he largely neglects. Darwin, of course, could only speculate about
the origin of these sex differences, which he tentatively reasoned to be
driven by the fact that insemination requires males to closely approach
females, thus selecting for an ``eagerness'' in males to seek out
females \citep{darwin1981}. But the brevity with which this, entirely
non-empirical, explanation is offered in The Descent of Man perhaps
speaks to the lack of confidence Darwin had in this line of reasoning.
In any case, Darwin's ideas on sexual selection and the associated coy
female-ardent male dichotomy largely languished in scientific no-man's
land until the latter half of the eponymous paradigm, Angus J. Bateman,
published an article titled ``Intra-sexual selection in Drosophila'' in
Heredity in 1948. Here, \citet{bateman1948} publishes two key findings
that would prove very influential for the science of mating systems: For
one, he found that mating success in males was significantly more
variable than mating success in females. This held both on a group
level, with 1 in 5 males failing to reproduce where for females, it was
only 1 in 50, and on the individual level, where number of offspring
varied much more between individual males than between individual
females. The second finding was that, additionally, reproductive success
of males seemed to increase linearly with number of mates - indicating
that males should strictly prefer more mating opportunities, within the
constraints of Bateman's experiments. Female reproductive success, on
the other hand, did not seem to benefit from additional mates beyond the
first one. This relationship between number of mates (or ``mating
success'') and reproductive success is referred to as Bateman's
gradient, the slope of which, after Bateman, is often posited to be
steeper for males than for females \citep{gerlach2012}. Collectively,
Bateman's results were collectively enshrined in modern sexual-selection
research as ``Bateman's principles'', seemingly offering and elegant
explanation for sex differences in sexual selection that had eluded
Darwin \citep{arnold1994}.

\hypertarget{explanations-for-marriage-patterns-in-humans}{%
\subsection{Explanations for marriage patterns in
humans}\label{explanations-for-marriage-patterns-in-humans}}

\hypertarget{polygyny-threshold-model}{%
\subsubsection{Polygyny threshold
model}\label{polygyny-threshold-model}}

\hypertarget{harem-defense-polygyny}{%
\subsubsection{Harem defense polygyny}\label{harem-defense-polygyny}}

\hypertarget{market-forces-sex-ratio-dependent-polygyny}{%
\subsubsection{Market forces / sex-ratio dependent
polygyny}\label{market-forces-sex-ratio-dependent-polygyny}}

\hypertarget{rival-wealth-induced-monogamy}{%
\subsubsection{Rival wealth-induced
monogamy}\label{rival-wealth-induced-monogamy}}

\hypertarget{cultural-group-selection}{%
\subsubsection{Cultural group
selection}\label{cultural-group-selection}}

  \bibliography{Dissertation.bib}

\end{document}
